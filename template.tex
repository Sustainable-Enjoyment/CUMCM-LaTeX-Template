%! Mode:: "TeX:UTF-8"
%! TEX program = xelatex
\PassOptionsToPackage{quiet}{xeCJK}
\documentclass[withoutpreface,notoc]{cumcmthesis}

% === 建议在此处加载 cls 文件未包含的、个人需要的宏包 ===
% 使用 natbib 的 super 选项实现上标引用
\usepackage[super,numbers,sort&compress]{natbib}
\usepackage[framemethod=TikZ]{mdframed} % 框架宏包
\usepackage{pdfpages} % 插入pdf页面

% === 建议在此处进行全局格式设置 ===
\usepackage{etoolbox}
% 将表格内的字体全局设置为小五号
\BeforeBeginEnvironment{tabular}{\zihao{-5}}

% === 自定义表格列类型 ===
\newcolumntype{C}{>{\centering\arraybackslash}X}
\newcolumntype{R}{>{\raggedleft\arraybackslash}X}
\newcolumntype{L}{>{\raggedright\arraybackslash}X}

\title{论文标题} % 论文标题

%%%%%%%%%%%%%%%%%%%%%%%%%%%%%%%%%%%%%%%%%%%%%%%%%%%%%%%%%%%%%
%% 正文
\begin{document}
	\maketitle
	\begin{abstract}
		摘要是论文的门面,需要高度概括你的工作。清晰地说明你们针对每个问题所采用的核心模型和关键算法,以及得到的主要结论和模型的亮点。
		
		\textbf{针对问题一,} 我们建立了...模型,通过...方法求解,得到了...的结果。
		
		\textbf{针对问题二,} 我们应用了...算法,分析了...,发现...。
		
		\textbf{针对问题三,} 我们创新性地提出了...,其优点在于...。
		
		\textbf{针对问题四,} ...
		
		最后,本文的模型具有较好的鲁棒性和可扩展性。
		
		\keywords{关键词\quad 关键词\quad 关键词\quad 关键词 \quad 关键词}
	\end{abstract}
	%%%%%%%%%%%%%%%%%%%%%%%%%%%%%%%%%%%%%%%%%%%%%%%%%%%%%%%%%%%%%
	
	% 国赛最终提交版本通常不包含目录页
	% \tableofcontents
	% \newpage
	
	%%%%%%%%%%%%%%%%%%%%%%%%%%%%%%%%%%%%%%%%%%%%%%%%%%%%%%%%%%%%%
	% 建议:第一节使用“问题重述”更符合竞赛范式
	\section{问题重述}
	在此部分简要重述赛题,厘清需要解决的关键问题点。这能向评委展示你对问题的理解是准确且深入的。
	
	%%%%%%%%%%%%%%%%%%%%%%%%%%%%%%%%%%%%%%%%%%%%%%%%%%%%%%%%%%%%%
	\section{模型假设}
	为简化问题,便于模型建立与求解,本文做出以下合理假设:
	\begin{itemize}[itemindent=2em]
		\item \textbf{假设1:} 这是一个非常重要的假设...
		\item \textbf{假设2:} 我们假设所有数据均是可靠且无误差的...
		\item \textbf{假设3:} 暂不考虑...等次要因素的影响。
	\end{itemize}
	
	%%%%%%%%%%%%%%%%%%%%%%%%%%%%%%%%%%%%%%%%%%%%%%%%%%%%%%%%%%%%%
	\section{符号说明}
	为方便阅读,本文使用的主要符号及其说明如\cref{tab:符号说明}所示。
	\begin{table}[H]
		\centering
		\caption{符号说明表} % 建议为所有图表添加标题
		\label{tab:符号说明}
		\begin{tabularx}{\textwidth}{CLC}
			\toprule
			符号 & 说明 & 单位 \\
			\midrule
			$m$ & 质量 & kg \\
			$g$ & 重力加速度 & m/s$^2$ \\
			$\alpha$ & 某个角度 & rad \\
			\bottomrule
		\end{tabularx}
	\end{table}
	
	%%%%%%%%%%%%%%%%%%%%%%%%%%%%%%%%%%%%%%%%%%%%%%%%%%%%%%%%%%%%%
	\section{问题一的模型建立与求解}
	\subsection{模型建立}
	根据题意和相关物理定律,我们可以建立...模型。
	\begin{lemma}
		这是一个引理...
	\end{lemma}
	\begin{definition}
		这是一个定义...
	\end{definition}
	
	% 建议:使用 \[...\] 代替 $$...$$
	\[
	E = mc^2
	\]
	
	这是一个需要引用的重要公式,见\cref{eq:爱因斯坦质能方程}。
	\begin{equation}
		\label{eq:爱因斯坦质能方程}
		E = mc^2
	\end{equation}
	
	下图 (\cref{fig:单图}) 展示了一个示例。
	\begin{figure}[ht]
		\centering
		\includegraphics[width=0.7\textwidth]{example.eps} % example.eps 是一个占位图
		\caption{单图示例}
		\label{fig:单图}
	\end{figure}
	
	这句话引用了文献\cite{司守奎2011数学建模算法与应用}。现在所有引用都将自动变为上标。
	
	\subsection{模型求解}
	为求解上述模型,我们设计了如下步骤:
	\begin{description}
		\item[Step 1:] 数据预处理...
		\item[Step 2:] 利用 MATLAB/Python 实现...算法...
		\item[Step 3:] 结果后处理与可视化...
	\end{description}
	
	\subsection{求解结果与分析}
	...
	
	%%%%%%%%%%%%%%%%%%%%%%%%%%%%%%%%%%%%%%%%%%%%%%%%%%%%%%%%%%%%%
	\section{问题二的模型建立与求解}
	\subsection{模型建立}
	对于问题二,我们考虑...因素,建立了...模型。如\cref{fig:双图}所示,其中\cref{fig:双图a}展示了...,\cref{fig:双图b}展示了...。
	
	\begin{figure}[ht]
		\centering
		\subcaptionbox{子图A的标题\label{fig:双图a}}
		{\includegraphics[width=.45\textwidth]{example.eps}}
		\hfill % 让两个子图尽可能分开
		\subcaptionbox{子图B的标题\label{fig:双图b}}
		{\includegraphics[width=.45\textwidth]{example.eps}}
		\caption{双子图示例}
		\label{fig:双图}
	\end{figure}
	
	% ... 后续问题类似 ...
	\section{模型的分析与检验}
	\subsection{灵敏度分析}
	...
	\subsection{误差分析}
	...
	%%%%%%%%%%%%%%%%%%%%%%%%%%%%%%%%%%%%%%%%%%%%%%%%%%%%%%%%%%%%%
	\section{模型的评价与推广}
	\subsection{模型的优点}
	\begin{itemize}[itemindent=2em]
		\item 优点1:模型结构简洁,物理意义明确。
		\item 优点2:算法效率高,求解速度快。
		\item 优点3:模型具有较好的通用性,可推广至...
	\end{itemize}
	\subsection{模型的缺点}
	\begin{itemize}[itemindent=2em]
		\item 缺点1:假设条件较强,在...情况下可能失效。
		\item 缺点2:未考虑...因素,与实际情况存在一定偏差。
	\end{itemize}
	
	%%%%%%%%%%%%%%%%%%%%%%%%%%%%%%%%%%%%%%%%%%%%%%%%%%%%%%%%%%%%%
	%% 参考文献
	% 建议:最终提交时删除 \nocite{*}
	% \nocite{*}
	\bibliographystyle{gbt7714-numerical} % 引用格式,符合国标
	\bibliography{ref} % 你的 .bib 文件名,这里假设是 ref.bib
	
	\newpage
	%%%%%%%%%%%%%%%%%%%%%%%%%%%%%%%%%%%%%%%%%%%%%%%%%%%%%%%%%%%%%
	%% 附录
	\begin{appendices}
		\section{支撑材料列表}
		\begin{table}[H]
			\centering
			\caption{附录文件列表}
			\label{tab:文件列表}
			\begin{tabularx}{\textwidth}{LL}
				\toprule
				文件名 & 功能描述 \\
				\midrule
				q1.m & 问题一核心 MATLAB 程序代码 \\
				q2.py & 问题二 Python 求解程序 \\
				data.xlsx & 本文使用的原始数据与处理结果 \\
				\bottomrule
			\end{tabularx}
		\end{table}
		
		\section{核心代码}
		\noindent\textbf{问题一 MATLAB 代码 (q1.m)}
		\lstinputlisting[language=matlab]{code/q1.m}
		
		\noindent\textbf{问题二 Python 代码 (q2.py)}
		\lstinputlisting[language=python]{code/q2.py}
		
	\end{appendices}
	
\end{document}